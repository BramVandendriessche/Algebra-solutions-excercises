\documentclass{article}
\usepackage[utf8]{inputenc}

\title{Toepassingen van Algebra oplossingen oefeningen}
\author{Pieter-Jan Coenen}
\date{December 2016}

\usepackage{natbib}
\usepackage{graphicx}
\usepackage{ dsfont }
\usepackage{enumitem}

\begin{document}

\maketitle

\section{Oefenzitting 1}
\subsection{Vraag 1}
Op $\mathds{R}$ defini�ren we de samenstellingswet $a\tau b = a+b+a^2 b^2$
\begin{enumerate}[label=(\alph*)]
\item Deze wet heeft een neutraal element. Welk?
\item Ze is niet associatief. Ga na !
\item Ze is commutatief. Waarom ?
\end{enumerate}

\textbf{Oplossing} cursus deel I blz 79
\begin{enumerate}[label=(\alph*)]
\item Voor de optelling is het neutraal element $0$ voor de vermenigvuldiging is het neutraal element $1$.
\\ We proberen eerst of het toepassen van de samenstellingswet op $0$ en $x \epsilon \mathds{R}$ terug resulteert in $x$.
    $$0\tau x = 0+x+0^2 x^2 = x$$
    $$x\tau 0 = x+0+x^2 0^2 = x$$
$0$ is dus het neutraal element.
\item Een samenstellingswet $\top$ is associatief als $x\top (y \top z) = (x \top y) \top z)$.\\
Met een tegenvoorbeeld aantonen dat dit hier niet het geval is volstaat dus.\\
Bijvoorbeeld:
    $$1\tau (2 \tau 3) = 1 \tau (2 + 3 + 4*9) = 1 \tau 41 = 1 + 41 + 1^2 41^2 = 1723 $$
    $$(1\tau 2) \tau 3 = (1 + 2 + 1*4) \tau 3 = 7 \tau 3 = 7 + 3 + 7^2 3^2 = 451$$
\item Ze is commutatief. Waarom ?
\end{enumerate}


\section{Conclusion}
``I always thought something was fundamentally wrong with the universe'' \citep{adams1995hitchhiker}

\bibliographystyle{plain}
\bibliography{references}
\end{document}
