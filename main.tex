\documentclass{article}
\usepackage[utf8]{inputenc}

\title{Toepassingen van Algebra oplossingen oefeningen}
\author{Pieter-Jan Coenen}
\date{December 2016}

\usepackage{natbib}
\usepackage{graphicx}
\usepackage{ dsfont }
\usepackage{ textcomp }
\usepackage{enumitem}
\usepackage{amsmath}
\usepackage{ amssymb }

\newcommand\arrows[3]{
	\quad
        \begin{array}{c}
        \ifx\relax#1\relax\else \xrightarrow{#1}\fi\\
        \ifx\relax#2\relax\else \xrightarrow{#2}\fi\\
        \ifx\relax#3\relax\else \xrightarrow{#3}\fi
        \end{array}
	\quad
}

\begin{document}

\maketitle

\section{Oefenzitting 1}
\subsection{Vraag 1}
Op $\mathds{R}$ definiëren we de samenstellingswet $a\tau b = a+b+a^2 b^2$
\begin{enumerate}[label=(\alph*)]
\item Deze wet heeft een neutraal element. Welk?
\item Ze is niet associatief. Ga na !
\item Ze is commutatief. Waarom ?
\end{enumerate}

\textbf{Oplossing} cursus deel I blz 79
\begin{enumerate}[label=(\alph*)]
\item Voor de optelling is het neutraal element $0$ voor de vermenigvuldiging is het neutraal element $1$.
\\ We proberen eerst of het toepassen van de samenstellingswet op $0$ en $x \in \mathds{R}$ terug resulteert in $x$.
    $$0\tau x = 0+x+0^2 x^2 = x$$
    $$x\tau 0 = x+0+x^2 0^2 = x$$
$0$ is dus het neutraal element.
\item Een samenstellingswet $\top$ is associatief $\Leftrightarrow \forall x,y \in A: x\top (y \top z) = (x \top y) \top z)$. (I blz 80)\\
We kunnen dus met een tegenvoorbeeld aantonen dat deze samenstellingswet niet associatief is.\\
Bijvoorbeeld:
    $$1\tau (2 \tau 3) = 1 \tau (2 + 3 + 4*9) = 1 \tau 41 = 1 + 41 + 1^2 41^2 = 1723 $$
    $$(1\tau 2) \tau 3 = (1 + 2 + 1*4) \tau 3 = 7 \tau 3 = 7 + 3 + 7^2 3^2 = 451$$
\item $\top$ is commutatief als $\Leftrightarrow \forall x,y \in A: x \top y  = y \top x $ (I blz. 80) \\
Deze samenstellingswet maakt enkel gebruik van de operatoren "+" en "*". Aangezien dat deze beide commutatief zijn zal ook de samenstellingswet commutatief zijn. \\
	$$\forall x,y \in A: x\tau y = x+y+x^2 y^2 = y+x+y^2 x^2 = y\tau x$$
\end{enumerate}


\subsection{Oefening 2}
Bewijs dat in $\mathds{R}^2\times\mathds{R}^2$ volgende relaties equivalentierelaties zijn:
	$$ G = \{((a, b),(c, d))|a^2 + b^2 = c^2 + d^2\}$$
	$$ H = \{((a, b),(c, d))|b-a = d-c\}$$
	$$ J = \{((a, b),(c, d))|a+b = c+d\}$$
Welke zijn de partities die hierdoor gedefinieerd worden? Welke partitie definieert $H \cap J$? \\ \\
\textbf{Oplossingsmethode} \\
Een relatie $R \subseteq A\times A$ is een equivalentierelatie (I blz.58) $\Leftrightarrow$
\begin{enumerate}
\item $R$ is reflexief $\Leftrightarrow$ elk element staat in relatie met zichzelf (I blz 59) :
	$$\forall x\in A : (x,x) \in R \text{ of } \forall x\in A : xRx$$
	Voorbeeld hiervalnis bijvoorbeeld de "equals-relatie" elk element is gelijk aan zichzelf $x=x$.
\item $R$ is symmetrisch $\Leftrightarrow$ de relatie in twee richtingen geldt (I blz 59) :
	$$(x,y) \in R \Rightarrow (y,x)\in R \text{ of } xRy\Rightarrow  yRx$$
	De "equals-relatie" is bijvoorbeeld symmetrisch want als $x=y \Rightarrow y=x$.
	De kleiner dan relatie is bijvoorbeeld niet symmetrisch wat als $x<y \nRightarrow y<x$.
\item $R$ is transitief $\Leftrightarrow$ de relatie kan doorgegeven worden (erfelijk is). (I blz 60) :
	$$(x,y) \in R\text{ en } (y,z) \in R \Rightarrow (x,z)\in R \text{ of } xRy\text{ en } yRz\Rightarrow  xRz$$
	De kleiner dan relatie is bijvoorbeeld transitief want als $x<y \text{ en } y < z \Rightarrow x < z$.
\end{enumerate}
\textbf{Oplossing voor G}
\begin{enumerate}
\item $G$ is reflexief want
	$$ \forall (x,y) \in\mathds{R}^2  \Rightarrow x^2 + y^2 = x^2 + y^2$$
	dus geldt dat 
	$$ \forall (x,y) \in\mathds{R}^2 : (x,y)G(x,y)$$
\item $G$ is symmetrisch want
	$$ \forall (x,y), (z,q) \in\mathds{R}^2: x^2 + y^2 = z^2 + q^2 \Rightarrow  z^2 + q^2 = x^2 + y^2 $$
	dus geldt dat 
	$$ (x,y)G(z,q) \Rightarrow (z,q)G(x,y) $$
\item $G$ is transitief want
	$$ \forall (x,y), (z,q),(v,w) \in\mathds{R}^2: (x^2 + y^2 = z^2 + q^2 \text{ \& } z^2 + q^2 = v^2 + w^2)  \Rightarrow  x^2 + y^2 = v^2 + w^2 $$
	dus geldt dat
	$$ (x,y)G(z,q) \text{ \& } (z,q)G(v,w)  \Rightarrow (x,y)G(v,w) $$
\end{enumerate}
Elke equivalentie relatie definieert een partitie (stelling 9.1 deel I blz 62).
Aangezien dat aan alle voorwaarden is voldaan, is $G$ een equivalentierelatie.\\ \\
\textbf{Oplossing voor H}
\begin{enumerate}
\item $H$ is reflexief want
	$$ \forall (x,y) \in\mathds{R}^2  \Rightarrow y-x = y-x$$
	dus geldt dat 
	$$ \forall (x,y) \in\mathds{R}^2 : (x,y)H(x,y)$$
\item $H$ is symmetrisch want
	$$ \forall (x,y), (z,q) \in\mathds{R}^2: y-x=q-z \Rightarrow  q-z=y-x $$
	dus geldt dat 
	$$ (x,y)H(z,q) \Rightarrow (z,q)H(x,y) $$
\item $H$ is transitief want
	$$ \forall (x,y), (z,q),(v,w) \in\mathds{R}^2: (y-x=q-z \text{ \& } q-z=w-v)  \Rightarrow  y-x=w-v $$
	dus geldt dat
	$$ (x,y)H(z,q) \text{ \& } (z,q)H(v,w)  \Rightarrow (x,y)H(v,w) $$
\end{enumerate}
Aangezien dat aan alle voorwaarden is voldaan, is $H$ een equivalentierelatie.\\ \\
\textbf{Oplossing voor J}
\begin{enumerate}
\item $J$ is reflexief want
	$$ \forall (x,y) \in\mathds{R}^2  \Rightarrow x+y=x+y$$
	dus geldt dat 
	$$ \forall (x,y) \in\mathds{R}^2 : (x,y)J(x,y)$$
\item $J$ is symmetrisch want
	$$ \forall (x,y), (z,q) \in\mathds{R}^2: x+y=z+q \Rightarrow  q+z=x+y $$
	dus geldt dat 
	$$ (x,y)J(z,q) \Rightarrow (z,q)J(x,y) $$
\item $J$ is transitief want
	$$ \forall (x,y), (z,q),(v,w) \in\mathds{R}^2: (x+y=z+q\text{ \& } z+q=v+w)  \Rightarrow  x+y=v+w $$
	dus geldt dat
	$$ (x,y)J(z,q) \text{ \& } (z,q)J(v,w)  \Rightarrow (x,y)J(v,w) $$
\end{enumerate}
Aangezien dat aan alle voorwaarden is voldaan, is $J$ een equivalentierelatie.\\ \\
\textbf{Oplossing bijvraag} \\ 
$H \cap J$ zijn dus alle koppels uit $\mathds{R}^2 $ die behoren tot zowel H als J.\\
Dit geeft de volgende formele beschrijving:
$$ H \cap J = \{((a, b),(c, d))|b-a = d-c \text{ \& } a+b = c+d\}$$
We kunnen dit verder uitwerken door dit in een stelsel te gieten:
$$
\begin{cases}
	b-a = d-c \\
	a+b = c+d
\end{cases}$$
Als we dit stelsel verder uitwerken krijgen we: 
$$
\begin{cases}
	b-a = d-c \\
	a+b = c+d
\end{cases}
=
\begin{cases}
	2b = 2d \\
	a+b = c+d
\end{cases}
=
\begin{cases}
	b = d \\
	a = c
\end{cases}
$$
Nu kunnen we $ H \cap J $ schrijven als:
	$$ H \cap J = \{((a, b),(c, d))| a = c \text{ \& } b = d\}$$
\subsection{Oefening 3}
Los het volgende stelsel op in (mod 7):
$$
\begin{cases}
	3x_1 - 2x_2 + 6x_3 = 4\\
	4x_1 - x_2 + x_3 = 0 \\
	2x_1 - x_2 + 2x_3 = -1
\end{cases}$$
\textbf{Oplossingsmethode}\\
Zie volledig uitgewerkt voorbeeld in deel I blz. 85. \\
Je kan beter geen deling gebruiken, want in sommige omstandigheden zorgt dit voor fouten. In plaats van een getal $x$ dus te delen door $x$ om een $1$ te bekomen moet je opzoek gaan naar een getal $y$ zodat $x*y=1$.\\
Bijvoorbeeld in modulo 5, om van $2$ naar $1$ te gaan doe je $2*3 = 6 \text{ mod } 5 = 1$.\\ \\
\textbf{Oplossing}\\
We zetten dit stelsel eerst om naar een matrix

\[
\left[
\begin{array}{ccc|c}
3 & -2 & 6 & 4 \\
4 & 1 & 1 & 0 \\
2 & 1 & 2 & -1 \\
\end{array}
\right]
\arrows{R_1 = R_1 * 5}{}{}
\left[
\begin{array}{ccc|c}
15 & -10 & 30 & 20 \\
4 & 1 & 1 & 0 \\
2 & 1 & 2 & -1 \\
\end{array}
\right]
\arrows{R_1 = R_1 \text{ mod } 5}{}{}
\left[
\begin{array}{ccc|c}
1 & 4 & 2 & 6 \\
4 & 1 & 1 & 0 \\
2 & 1 & 2 & -1 \\
\end{array}
\right]
\]
\[
\arrows{}{R_2 = R_2 - 4*R_1}{R_3 = R_3 - 2*R_1}
\left[
\begin{array}{ccc|c}
1 & 4 & 2 & 6 \\
0 & -15 & -7 & -24 \\
0 & -7 & -2 & -13 \\
\end{array}
\right]
\arrows{}{R_2 = R_2 \text{ mod } 7}{R_3 =R_3 \text{ mod } 7}
\left[
\begin{array}{ccc|c}
1 & 4 & 2 & 6 \\
0 & 6 & 0 & 4 \\
0 & 0 & 5 & 1\\
\end{array}
\right]
\]
\[
\arrows{}{R_2 = R_2 * 6}{R_3 = R_3 * 3}
\left[
\begin{array}{ccc|c}
1 & 4 & 2 & 6 \\
0 & 1 & 0 & 3 \\
0 & 0 & 1 & 3\\
\end{array}
\right]
\]
Dit resulteert in
$$
\begin{cases}
	x_1 + 4x_2 + 2x_3 = 6\\
	x_2 = 3 \\
	x_3 = 3
\end{cases}
=
\begin{cases}
	x_1 + 18 \text{ (mod }7) = x_1 + 4 = 6\\
	x_2 = 3 \\
	x_3 = 3
\end{cases}
=
\begin{cases}
	x_1 = 2\\
	x_2 = 3 \\
	x_3 = 3
\end{cases}
$$
\subsection{Oefening 4}
Bepaal de isometrieën van een gelijkzijdige driehoek. Stel voor deze isometrieën de bewerkingstabel op, onder de samenstellingswet ◦.

\end{document}
